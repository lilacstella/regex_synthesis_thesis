%%%%%%%%%%%%%%%%%%%%%%%%%%%%
%  Template code for the Undergraduate Research Scholars thesis program starting, updated by Undergraduate Research Scholars program staff. Version 6.0. Last Updated: Fall 2024
%  Modified by Tawfik Hussein from the template code for TAMU Theses and Dissertations starting Spring 2018, authored by Sean Zachary Roberson. Version 3.17.09.
%
%%%%%%%%%%%%%%%%%%%%%%%%%%%%
%%%%%%%%%%%%%%%%%%%%%%%%%%%%
% ABSTRACT (REQUIRED PAGE - DO NOT REMOVE)
%%%%%%%%%%%%%%%%%%%%%%%%%%%%

%_________(0)_________
% % Do not modify. This is the required page heading. 

\chapter*{\large\bf ABSTRACT}

%_________(1)_________
% Do not modify. This adds the Abstract section to the Table of Contents.

\addcontentsline{toc}{chapter}{ABSTRACT} 

%_________(2)_________
% Do not modify. This controls the vertical spacing between the word "Abstract" and the thesis title.

\vspace{-1em}

%_________(3)_________
% Do not modify. This section single-spaces and centers the thesis title.

\begin{singlespace}

\begin{center}

%_________(4)_________
% Modifications Needed!
% Enter your thesis title in Title Case exactly as it appears on the Title page.

\tamumanuscripttitle
\vspace{3em}

%_________(5)_________
% Modifications Needed!
% Enter your name or team names exactly as they appear in other sections (spelling, order, superscripts, etc.). Enter department affiliations. Note that your major may be different than your department name.

\tamufullname\\
Department of Computer Science and Engineering\\
Texas A\&M University
\vspace{3em}

%_________(6)_________
% Modifications Needed!
% Enter the name(s) and department(s) of your advisor(s). Don't forget to include "Dr." if they have a PhD. If they don't have a PhD, use their post-nominal letters. 
% The template includes two advisor slots. If you only have one, delete the secondary advisor slot. However, if you have two or more advisors, be sure to include them on the Title page exactly as you list them here. If both advisors are in the same department, you can place their names on the same line. However, if they belong to different departments, give them their own sections. If you have more than two advisors, copy and paste one of the sections below text below and modify accordingly.

Faculty Research Advisor: \tamuadvisor
\\
Department of Computer Science and Engineering
\\
Texas A\&M University

\vspace{3em}

% Faculty Research Advisor: [Type Dr. Full Name of Secondary Faculty Advisor OR remove line]
% \\
% {\colorbox{Yellow}{[Choose an item:} Department/s\colorbox{Yellow}{]}} of [Type Secondary Faculty Advisor Department OR remove line]
% \\
% Texas A\&M University
% \vspace{1em}
\end{center}
\end{singlespace}

%_________(7)_________
% Do not modify. This section starts your thesis at page 1 on the Abstract page (required).

\pagestyle{plain}
\pagenumbering{arabic}
\setcounter{page}{1}

%_________(8)_________
% Modifications Needed!
% Use the Double /indent command to properly indent all paragraphs. Enter your abstract text.


\indent\indent This thesis explores the problem of test-driven program synthesis in the domain of regular expressions, a class of formal languages that describes a foundational class of problems in computer science, and occassionally frustrating to write. Despite their widespread use in tasks like data validation and pattern matching, regular expressions remain a pain point for many developers due to their terse syntax and limited readability. We investigate the capacity of three distinct synthesis paradigms—heuristic search (AlphaRegex), procedural inference (L* algorithm), and large language model generation (ChatGPT)—to generate correct, minimal, and human-readable regex from a set of positive and negative examples.

\indent\indent We explore the three different methods using 25 regex generation problems of varying complexity and evaluates each method across dimensions of correctness, efficiency, and complexity. AlphaRegex uses weighted explicit search and formal approximations to prune the search space. The L* algorithm leverages query-based learning minimal finite state machines. LLMs rely on whether similar patterns or prompt structures have been encountered during training, generating responses based on statistical correlations rather than formal correctness guarantees.

\indent\indent We find that each approach has clear strengths: AlphaRegex is systematic but brittle, L* provides principled correctness under guided feedback, and LLMs deliver an approximate answer that typically resemble a real solution. Preliminary experiments in integrated systems suggest promising directions for future work. Ultimately, this thesis highlights both the challenges and potential in aligning human intent with machine-generated logic, using regex as a microcosm for broader program synthesis goals.
 

\pagebreak{}
