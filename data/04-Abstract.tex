%%%%%%%%%%%%%%%%%%%%%%%%%%%%
%  Template code for the Undergraduate Research Scholars thesis program starting, updated by Undergraduate Research Scholars program staff. Version 6.0. Last Updated: Fall 2024
%  Modified by Tawfik Hussein from the template code for TAMU Theses and Dissertations starting Spring 2018, authored by Sean Zachary Roberson. Version 3.17.09.
%
%%%%%%%%%%%%%%%%%%%%%%%%%%%%
%%%%%%%%%%%%%%%%%%%%%%%%%%%%
% ABSTRACT (REQUIRED PAGE - DO NOT REMOVE)
%%%%%%%%%%%%%%%%%%%%%%%%%%%%

%_________(0)_________
% % Do not modify. This is the required page heading. 

\chapter*{\large\bf ABSTRACT}

%_________(1)_________
% Do not modify. This adds the Abstract section to the Table of Contents.

\addcontentsline{toc}{chapter}{ABSTRACT} 

%_________(2)_________
% Do not modify. This controls the vertical spacing between the word "Abstract" and the thesis title.

\vspace{-1em}

%_________(3)_________
% Do not modify. This section single-spaces and centers the thesis title.

\begin{singlespace}

\begin{center}

%_________(4)_________
% Modifications Needed!
% Enter your thesis title in Title Case exactly as it appears on the Title page.

\tamumanuscripttitle
\vspace{3em}

%_________(5)_________
% Modifications Needed!
% Enter your name or team names exactly as they appear in other sections (spelling, order, superscripts, etc.). Enter department affiliations. Note that your major may be different than your department name.

\tamufullname\\
Department of Computer Science and Engineering\\
Texas A\&M University
\vspace{3em}

%_________(6)_________
% Modifications Needed!
% Enter the name(s) and department(s) of your advisor(s). Don't forget to include "Dr." if they have a PhD. If they don't have a PhD, use their post-nominal letters. 
% The template includes two advisor slots. If you only have one, delete the secondary advisor slot. However, if you have two or more advisors, be sure to include them on the Title page exactly as you list them here. If both advisors are in the same department, you can place their names on the same line. However, if they belong to different departments, give them their own sections. If you have more than two advisors, copy and paste one of the sections below text below and modify accordingly.

Faculty Research Advisor: \tamuadvisor
\\
Department of Computer Science and Engineering
\\
Texas A\&M University

\vspace{3em}

% Faculty Research Advisor: [Type Dr. Full Name of Secondary Faculty Advisor OR remove line]
% \\
% {\colorbox{Yellow}{[Choose an item:} Department/s\colorbox{Yellow}{]}} of [Type Secondary Faculty Advisor Department OR remove line]
% \\
% Texas A\&M University
% \vspace{1em}
\end{center}
\end{singlespace}

%_________(7)_________
% Do not modify. This section starts your thesis at page 1 on the Abstract page (required).

\pagestyle{plain}
\pagenumbering{arabic}
\setcounter{page}{1}

%_________(8)_________
% Modifications Needed!
% Use the Double /indent command to properly indent all paragraphs. Enter your abstract text.


\indent\indent This thesis investigates the problem of program synthesis in the constrained domain of regular expression (regex) generation. We compare three distinct methodologies: AlphaRegex, a heuristic-guided search-based synthesizer; the L* algorithm, a procedural learner rooted in automata theory; and large language models (LLMs), which generate regex through few-shot prompting. Our evaluation framework assesses each approach on correctness, efficiency, and complexity, using a diverse benchmark set spanning easy to hard regex synthesis tasks.

\indent\indent We find that AlphaRegex benefits from tunable cost heuristics that allow control over the balance between simplicity and runtime but suffers in scalability on complex patterns. The L* algorithm reliably produces correct regex through interaction with a teacher but can become rigid and verbose due to its dependence on counterexamples and DFA construction. LLMs perform surprisingly well on simple problems that resemble training data patterns but degrade with compositional complexity and struggle with formal constraints. Additionally, we analyze AlphaRegex under varying heuristic weight configurations and explore the limitations and benefits of hybrid (or chained) synthesis pipelines.

\indent\indent This work contributes a detailed experimental comparison across methodologies, a public implementation of each, and reflections on what it means to "ask good questions" when synthesizing code. It lays groundwork for future synthesis tools that integrate heuristic, procedural, and neural strategies, offering insights into usability, test-driven development, and scalable human-in-the-loop programming workflows.

\pagebreak{}
